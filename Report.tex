\documentclass[a4paper,12pt]{article}
\usepackage[english]{babel}
\usepackage{blindtext}
\usepackage[a4paper,inner=1.5cm,outer=3cm,top=2cm,bottom=3cm]{geometry}
\usepackage{amsmath}
\usepackage{graphicx}
\usepackage{subfigure}
\usepackage{multirow}
\usepackage{array}
\usepackage{booktabs}
\usepackage{setspace}

\begin{document}
\onehalfspacing
\section{Description}
  \subsection{Introduction}
   	 Stress echocardiography is a test which uses ultrasound imaging to show how well your heart muscles are working to pump blood to your body. In the project study, we are interested the effects of the test on predicting cardiac events, which are separated into four categories: myocardial infarction (newMI), revascularization by percutaneous transluminal coronary angioplasty (newPTCA), coronary artery bypass grafting surgery (newCABG), cardiac death (death).\\


	First, we develop the basic test such as two-sample z-test for difference in mean, and chi-square test for independence. And then we established logistic regression model. Since there are several continuous predictors in the model, we developed the Hosmer-Lemeshow "goodness of fit" test. Finally the CART analysis is established to find the variables which can predict future cardiac events.\\
\textbf{Key words:}	  Stress echocardiography, cardiac events, logistic regression, CART.
  \subsection{Data resource}
    \begin{table}[htbp]
      \centering
      \caption{List of Predictors}
        \begin{tabular}{ll|ll}
        \toprule[1.5pt]        Variable & Description      & Variable     &Description  \\
        \midrule
        bhr   & Basal  Heart Rate & age   & Age \\
        basebp & Basal Blood Pressure & gender & Gender (male=0) \\
        basedp & Basal Double Product & baseEF & Baseline Cardiac EF \\
              &  (bhr*basebp) & dobEF & EF on Dobutamine \\
        pkhr  & Peak Heart Rate & chestpain  & Experienced Chest Pain \\
        sbp   & Systolic Blood Pressure & posECG & Signs of Heart attack on ECG \\
        dp    & Double Product(pkhr*sbp) & equivecg & ECQ is Equivocal \\
        dose  & Dose of Dobutamine Give & restwma & Wall Motion Anamoly on \\
        maxhr & Maximum Heart Rate &       &  Echocardiogram \\
        \%mphr(b) & \% of Maximum Predicted Heart & posSE  & Positive SE  \\
              & Rate Achieved by Patient & hxofHT & History of Hypertension \\
        mbp   & Maximum Blood Pressure & hxofdm & History of Diabetes \\
        dpmaxdo & Double Product on Maximum & hxofcig & History of Smoking \\
              &  Dobutamine Dose & hxofMI & History of Heart Attack \\
        dobdose & Dobutamine Dose at Which  & hxofPTCA & History of Angioplasty \\
              & Maximum DP Occured & hxofCABG & History of Bypass Surgery \\
        \bottomrule[1.5pt]
        \end{tabular}%
      \label{LoV}%
    \end{table}%
   In Table \ref{LoV}, EF is short for EJECTION FRACTION, a very common used measure of the heart's pumping efficiency. SE is short for STRESS ECHOCARDIOGRAM. As for the result table, 0 means the patient DID experienced corresponding event and 1 means they DID NOT.
   % Table generated by Excel2LaTeX from sheet 'Sheet2'
   \begin{table}[htbp]
     \centering
     \caption{List of Response}
       \begin{tabular}{ll}
       \toprule[1.5pt]
       Response &  \\
       \midrule
       newMI & New Myocardial Infration or Heart Attack \\
       newPTCA & Recent Angioplasty \\
       newCABG & Recent Bypass Surgery \\
       death & The Patient Died \\
       any event & newMI or newPTCA or newCABG or Death \\
       \bottomrule[1.5pt]
       \end{tabular}%
     \label{LoR}
   \end{table}%
\section{Generalized Linear Model}
  \subsection{Two-sample Z-test}
  Choosing from all the numerical variables, we divided them into 3 groups based on similarity. The first Group contains all items related to Heart Rate, and the second includes Blood Pressure, and the rest three are all about Product(Heart Rate$\times$Blood Pressure).
    \begin{table}[htbp]
      \centering
      \caption{The Result of Two-sample Z-test}
        \begin{tabular}{lrrlrrr}
        \toprule[1.5pt]
              & Mean  & Variance &       & Mean  & Variance & Z-value \\
        \midrule
        bhr   & 75.29 & 237.63 &  pkhr & 120.55 & 509.31 & -39.12 \\
        bhr   & 75.29 & 237.63 &  maxhr & 119.37 & 479.92 & -38.87 \\
        maxhr & 119.37 & 479.92 &  pkhr & 120.55 & 509.31 & 0.89 \\
        \midrule
        sbp   & 146.92 & 1334.41 &  basebp & 135.32 & 431.40 & -6.52 \\
        mbp   & 156.00 & 1005.25 &  basebp & 135.32 & 431.40 & -12.89 \\
        mbp   & 156.00 & 1005.25 &  sbp  & 146.92 & 1334.41 & -4.44 \\
        \midrule
        dp    & 17633.84 & 27253975.02 &  basedp & 10181.31 & 6655106.80 & -30.23 \\
        dpmaxdo & 18549.88 & 24024010.35 &  basedp & 10181.31 & 6655106.80 & -35.69 \\
        dpmaxdo & 18549.88 & 24024010.35 &  dp   & 17633.84 & 27253975.02 & -3.02 \\
        \bottomrule[1.5pt]
        \end{tabular}%
      \label{ZT}%
    \end{table}%
    From Table \ref{ZT}, we can find that maxhr \& pkhr can be treated as having same mean. In order to eliminate unnecessary variables, and make sure the model is not effected, we checked correlation coefficient also and the figure is displayed in Figure \ref{corr}.
      \begin{figure}[htbp]
        \centering
        \includegraphics[height=4in]{ccor}
        \caption{Correlation between Heart Rated Related Variables}
        \label{corr}
      \end{figure}
    Based on information we got from Table \ref{ZT} and Figure \ref{corr}, we think pkhr and maxhr can be regarded as the same for the high corr(0.954) and same mean. That means we would not choose both of them Simultaneously.
  \subsection{Chi-square Test}
  For the rest variables, category ones, we applied Chi-square Test to them in order to check the independence. What should point out here is items such as chestpain is whether the patient experienced the chest pain before, not after they took the medicine. and the patients' history of all related items were calculated here: Hypertension, Diabetes, Smoking, Heart Attack, Angioplasty, Bypass Surgery.\\
  Here we are testing the independence between category predictor and responds, and we will take gender as an example in detail and give a summary in Table \ref{CT}.
    % Table generated by Excel2LaTeX from sheet 'Sheet1'
    \begin{table}[htbp]
      \centering
      \caption{Gender VS Any Event}
        \begin{tabular}{c|cc}
        \toprule[1.5pt]
        gender & positive & negative \\
        \midrule
        0     & 177   & 43 \\
        1     & 292   & 46 \\
        \bottomrule[1.5pt]
        \end{tabular}%
      \label{GVE}%
    \end{table}%
  Based on Table \ref{GVE}, we can calculated that the respect $\chi^2 = 11.9148$ and the corresponding $p=0.0006$, which means the relationship between gender and any event is not independent.
    \begin{table}[htbp]
      \centering
      \caption{The Result of Chi-sq Test}
        \begin{tabular}{lrrlrr}
        \toprule[1.5pt]
              & Chi-sq & P-value &       & Chi-sq & P-value \\
        \midrule
        gender  & 11.9148 & 0.0006 & hxofHT  & 6.0665 & 0.0138 \\
        chestpain  & 3.0739 & 0.0796 & hxofdm  & 78.3402 & 0.0000 \\
        posECG  & 0.1414 & 0.7069 & hxofMI  & 3.9594 & 0.0466 \\
        equivecg  & 7.5186 & 0.0061 & hxofPTCA  & 1.6385 & 0.2005 \\
        restwma  & 0.7213 & 0.3957 & hxofCABG  & 36.6545 & 0.0000 \\
        posSE  & 82.5844 & 0.0000 &       &       &  \\
        \bottomrule[1.5pt]
        \end{tabular}%
      \label{CT}%
    \end{table}%
 Form the summary of result, Table \ref{CT}, the independent predictors are: gender, equivecg, posSE, hxofHT, hxofdm, hxofMI and hxofCABG.

 \subsection{Logistic Regression Analysis}
 To fit the best model, we begin our analyze with full model and without any iteration items, which means we have 25 variables at beginning and than back forward. We eliminated the most insignificant item one by one and run a regression model every and each time.\\
 We started with predictor dobdose, which had a p-value equals to 0.9747, the most insignificant one and went all the way down until all the remain item is significant, which is present in Table \ref{M1}.
   \begin{table}[htbp]
     \centering
     \caption{Summary of Item Eliminated for Model 1}
       \begin{tabular}{lcclcc}
       \toprule[1.5pt]
       Item elimated  & p-value & AIC   & Item elimated  & p-value & AIC \\
       \midrule
       dobdose & 0.9747 & 439.83 & chestpain & 0.4248 & 426.38 \\
       hxofPTCA & 0.7619 & 437.84 & baseEF & 0.4161 & 425.01 \\
       hxofcig & 0.6423 & 435.93 & dose  & 0.3339 & 423.67 \\
       equivecg & 0.6164 & 434.14 & basebp & 0.1458 & 422.60 \\
       hxofCABG & 0.5937 & 432.39 & bhr   & 0.7754 & 422.74 \\
       gender & 0.5315 & 430.68 & hxofdm & 0.0576 & 420.57 \\
       pkhr  & 0.4632 & 427.81 & hxofMI & 0.1230 & 422.16 \\
       \midrule
             &       &       & $\mathbf{Final}$ &       & $\mathbf{422.51}$ \\
       \bottomrule[1.5pt]
       \end{tabular}%
     \label{M1_E}%
   \end{table}%

   \begin{table}[htbp]
     \centering
     \caption{Summary of Model 1}
       \begin{tabular}{lrrrrl}
       \toprule[1.5pt]
             & Estimate & Std. Error & z value & P-Value &  \\
       \midrule
       (Intercept) & -6.6853 & 2.2654 & -2.95 & 0.0032 & ** \\
       dp    & 0.0003 & 0.0001 & 2.07  & 0.0385 & * \\
       sbp   & -0.0292 & 0.0153 & -1.91 & 0.0565 & . \\
       dpmaxdo & -0.0006 & 0.0002 & -2.60 & 0.0093 & ** \\
       mphr  & 0.0912 & 0.0436 & 2.09  & 0.0363 & * \\
       mbp   & 0.0733 & 0.0283 & 2.59  & 0.0096 & ** \\
       age   & -0.0526 & 0.0259 & -2.03 & 0.0421 & * \\
       dobEF & 0.0351 & 0.0106 & 3.30  & 0.0010 & *** \\
       posECG & 0.8059 & 0.3255 & 2.48  & 0.0133 & * \\
       restwma & 0.8547 & 0.3524 & 2.43  & 0.0153 & * \\
       posSE & 0.9460 & 0.2808 & 3.37  & 0.0008 & *** \\
       hxofHT & 0.8002 & 0.3324 & 2.41  & 0.0161 & * \\
       \bottomrule[1.5pt]
       \end{tabular}%
     \label{M1}%
   \end{table}%
 What does matter us here was that sbp, had a p-value 0.0565, which is slightly greater than 0.05. If we delete that one, we need to remove a lot of other items also, so we developed our second model as below.
 % Table generated by Excel2LaTeX from sheet 'Sheet1'
 \begin{table}[tbp]
   \centering
   \caption{Summary of Item Eliminated for Model 2}
     \begin{tabular}{lcclcc}
     \toprule[1.5pt]
     Item elimated  & p-value & AIC   & Item elimated  & p-value & AIC \\
     \midrule
     dobdose & 0.9747 & 439.83 & bhr   & 0.1894 & 422.60 \\
     hxofPTCA & 0.7619 & 437.84 & basebp & 0.4940 & 422.35 \\
     hxofcig & 0.6423 & 435.93 & hxofMI  & 0.1046 & 420.57 \\
     equivecg & 0.6164 & 434.14 & hxofdm  & 0.0671 & 421.17 \\
     hxofCABG & 0.5937 & 432.39 & sbp   & 0.0565 & 422.51 \\
     gender & 0.5315 & 430.68 & dp    & 0.4852 & 424.43 \\
     pkhr  & 0.4632 & 427.81 & dpmaxdo & 0.0884 & 422.91 \\
     chestpain & 0.4248 & 426.38 & age   & 0.3684 & 423.99 \\
     baseEF & 0.4161 & 425.01 & mbp   & 0.2675 & 422.81 \\
     dose  & 0.3339 & 423.67 & mphr  & 0.26556 & 422.06 \\
     \midrule
           &     &     & $\mathbf{Final}$ &       & $\mathbf{421.31}$ \\
     \bottomrule[1.5pt]
     \end{tabular}
   \label{M2_E}
 \end{table}%

  % Table generated by Excel2LaTeX from sheet 'Sheet1'
  \begin{table}[h]
    \centering
    \caption{Summary of Model 2}
      \begin{tabular}{lrrrrl}
      \toprule[1.5pt]
            & Estimate & Std. Error & z value & P-value &  \\
      \midrule
      (Intercept) & -2.31417 & 0.68591 & -3.374 & 0.000741 & *** \\
      dobEF & 0.03619 & 0.01035 & 3.495 & 0.000474 & *** \\
      posECG & 0.78025 & 0.31761 & 2.457 & 0.014025 & * \\
      restwma & 0.85254 & 0.34734 & 2.454 & 0.01411 & * \\
      posSE & 0.88881 & 0.27345 & 3.25  & 0.001153 & ** \\
      hxofHT & 0.8384 & 0.31761 & 2.64  & 0.008297 & ** \\
      \bottomrule[1.5pt]
      \end{tabular}
    \label{M2}
  \end{table}%
  Compared Table \ref{M2_E} to Table \ref{M1_E}, our final AIC actually had a narrowly decrement, but almost the same. While if we took the number of predictors into consideration, we can find that the second model is much better than the first one with only 5 items left, comparing between Table \ref{M1} and \ref{M2}. \\

So we chose the second model as our regression model, which has only five item left: dobEF, posECG, restwma, posSE, and hxofHT. Here, we can find that the most common used variables, such as gender and age, have almost no influence on the experiments results. The most significant one is dobEF, which means Ejection Fraction on Dobutamine. It seems that Dobutamine has some influence on the result.
  \subsection{Goodness of Fit Test}
  Since our model has both numerical and category variables, our model should be treated as logistic regression model, which means we can use Hosmer-Lemeshow test for the model to check the goodness of fit.
  % Table generated by Excel2LaTeX from sheet 'Sheet3'
  \begin{table}[htbp]
    \centering
    \caption{Result of H-L Test}
      \begin{tabular}{cccc}
      \toprule
            & $ {\chi}^2$ & df    & P-value \\
      \midrule
      Model 1 & 4.1749 & 8     & 0.8410 \\
      Model 2 & 4.8442 & 8     & 0.7741 \\
      \bottomrule
      \end{tabular}%
    \label{HL}%
  \end{table}\\
  For Table \ref{HL}, our both models have much greater p-values than 0.05, which means the they both are good enough for the fitness.

\section{CART Analysis}
  Upon the purpose of study about finding the best helpful measurement during the stress echocardiography in predicting whether or not a patient suffered a cardiac event over the next year, the Classification and Regression Tree (CART) here performs well on multivariate analysis. CART model identified variables that allow patients to be successively split into subsets with best predicted future cardiac events.\\
  The tree can be generated through the rpart package. The result of some steps are illustrated in Figure \ref{Tree 1}
  \subsection{Grow the Tree}
      \begin{figure}[t]
        \centering
        \includegraphics[width=16cm,height=9cm]{Tree}
        \caption{Regression Tree for Cardiac}
        \label{Tree 1}
      \end{figure}
      \begin{figure}[htbp]
        \centering
        \includegraphics[width=14cm,height=7cm]{Tree2}
        \caption{Summary Info of Regression Tree}
        \label{Tree 2}
      \end{figure}
  From the left graph in Figure \ref{Tree 2}, the R-square for X Relative increases and then drops as size of tree increases. It indicates the same meaning as the right graph. Over-fitting occurs due to a single large decision tree.\\
  We can use this information from Figure \ref{Tree 2} to decide how complex the tree needs to be.
  \subsection{Prune Back the Tree}
  Select a tree size by Leaving out one of the subset to minimize the cross-validated error. We have the result in Figure \ref{Tree 3}.
    \begin{figure}[htbp]
      \centering
      \includegraphics[width=14cm,height=10cm]{Tree3}
      \caption{Prune Back Tree}
      \label{Tree 3}
    \end{figure}\\
  CART illustrating likelihood of a patient having a cardiac event. The total number of patients, N=558(the pie circle on the top). The area of each ie is proportional to N in each subgroup. The shaded area of each pie is the percentage of patients in that group who had a cardiac event.
  \subsection{CART Result}
  Classification and regression tree analysis (Figure \ref{Tree 3}) indicates that the best single predictor of a cardiac event was dobEF. Thus, the ejection fraction on dobutamine(dobEF) is the first branch of the tree used to predict future cardiac events. For those 77 with dobEF less than 52.5\%(the left branch of the tree)the next most useful predictor was patient has history of hypertension. The third branch on the left side of the tree was basal blood pressure. The fourth branch on this side of the tree was age. Twenty-four patients had dobEF less than 52.5\%, the history of hypertension,  basal blood pressure$ < $142, and age$ < $77. Of these 24 patients, 75\% had a cardiac event compared with the 16\% event rate of the entire patient population.\\
  The basebp is continuously as the critical predictive factor at the fourth branch in the 24 patients with dobEF$ < $52.5\% , the history of hypertension and the basebp$ > $142. In this category, 67\% had an event if their basebp$\ge$163.5, versus a 13\% event rate if the basebp$<$163.5.
\section{Summary}
  \subsection{Conclusion}
  Applying two methods, we still get the same result that dobEF alone is the best single predictor. And also, the History of Hypertension plays the second important role in the experiments. Like mentioned before, the most common used index such as gender, age and heart rate, in realistic, have no significant influence here.

  \subsection{Package Used}
  lme4, rpart, ResourceSelection, ggplot, Matrix, etc.
\end{document} 
